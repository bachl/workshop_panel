\PassOptionsToPackage{unicode=true}{hyperref} % options for packages loaded elsewhere
\PassOptionsToPackage{hyphens}{url}
%
\documentclass[]{book}
\usepackage{lmodern}
\usepackage{amssymb,amsmath}
\usepackage{ifxetex,ifluatex}
\usepackage{fixltx2e} % provides \textsubscript
\ifnum 0\ifxetex 1\fi\ifluatex 1\fi=0 % if pdftex
  \usepackage[T1]{fontenc}
  \usepackage[utf8]{inputenc}
  \usepackage{textcomp} % provides euro and other symbols
\else % if luatex or xelatex
  \usepackage{unicode-math}
  \defaultfontfeatures{Ligatures=TeX,Scale=MatchLowercase}
\fi
% use upquote if available, for straight quotes in verbatim environments
\IfFileExists{upquote.sty}{\usepackage{upquote}}{}
% use microtype if available
\IfFileExists{microtype.sty}{%
\usepackage[]{microtype}
\UseMicrotypeSet[protrusion]{basicmath} % disable protrusion for tt fonts
}{}
\IfFileExists{parskip.sty}{%
\usepackage{parskip}
}{% else
\setlength{\parindent}{0pt}
\setlength{\parskip}{6pt plus 2pt minus 1pt}
}
\usepackage{hyperref}
\hypersetup{
            pdftitle={Mini-Workshop Panel Data Analysis},
            pdfauthor={Marko Bachl (mit Material von Michael Scharkow)},
            pdfborder={0 0 0},
            breaklinks=true}
\urlstyle{same}  % don't use monospace font for urls
\usepackage{color}
\usepackage{fancyvrb}
\newcommand{\VerbBar}{|}
\newcommand{\VERB}{\Verb[commandchars=\\\{\}]}
\DefineVerbatimEnvironment{Highlighting}{Verbatim}{commandchars=\\\{\}}
% Add ',fontsize=\small' for more characters per line
\usepackage{framed}
\definecolor{shadecolor}{RGB}{248,248,248}
\newenvironment{Shaded}{\begin{snugshade}}{\end{snugshade}}
\newcommand{\AlertTok}[1]{\textcolor[rgb]{0.94,0.16,0.16}{#1}}
\newcommand{\AnnotationTok}[1]{\textcolor[rgb]{0.56,0.35,0.01}{\textbf{\textit{#1}}}}
\newcommand{\AttributeTok}[1]{\textcolor[rgb]{0.77,0.63,0.00}{#1}}
\newcommand{\BaseNTok}[1]{\textcolor[rgb]{0.00,0.00,0.81}{#1}}
\newcommand{\BuiltInTok}[1]{#1}
\newcommand{\CharTok}[1]{\textcolor[rgb]{0.31,0.60,0.02}{#1}}
\newcommand{\CommentTok}[1]{\textcolor[rgb]{0.56,0.35,0.01}{\textit{#1}}}
\newcommand{\CommentVarTok}[1]{\textcolor[rgb]{0.56,0.35,0.01}{\textbf{\textit{#1}}}}
\newcommand{\ConstantTok}[1]{\textcolor[rgb]{0.00,0.00,0.00}{#1}}
\newcommand{\ControlFlowTok}[1]{\textcolor[rgb]{0.13,0.29,0.53}{\textbf{#1}}}
\newcommand{\DataTypeTok}[1]{\textcolor[rgb]{0.13,0.29,0.53}{#1}}
\newcommand{\DecValTok}[1]{\textcolor[rgb]{0.00,0.00,0.81}{#1}}
\newcommand{\DocumentationTok}[1]{\textcolor[rgb]{0.56,0.35,0.01}{\textbf{\textit{#1}}}}
\newcommand{\ErrorTok}[1]{\textcolor[rgb]{0.64,0.00,0.00}{\textbf{#1}}}
\newcommand{\ExtensionTok}[1]{#1}
\newcommand{\FloatTok}[1]{\textcolor[rgb]{0.00,0.00,0.81}{#1}}
\newcommand{\FunctionTok}[1]{\textcolor[rgb]{0.00,0.00,0.00}{#1}}
\newcommand{\ImportTok}[1]{#1}
\newcommand{\InformationTok}[1]{\textcolor[rgb]{0.56,0.35,0.01}{\textbf{\textit{#1}}}}
\newcommand{\KeywordTok}[1]{\textcolor[rgb]{0.13,0.29,0.53}{\textbf{#1}}}
\newcommand{\NormalTok}[1]{#1}
\newcommand{\OperatorTok}[1]{\textcolor[rgb]{0.81,0.36,0.00}{\textbf{#1}}}
\newcommand{\OtherTok}[1]{\textcolor[rgb]{0.56,0.35,0.01}{#1}}
\newcommand{\PreprocessorTok}[1]{\textcolor[rgb]{0.56,0.35,0.01}{\textit{#1}}}
\newcommand{\RegionMarkerTok}[1]{#1}
\newcommand{\SpecialCharTok}[1]{\textcolor[rgb]{0.00,0.00,0.00}{#1}}
\newcommand{\SpecialStringTok}[1]{\textcolor[rgb]{0.31,0.60,0.02}{#1}}
\newcommand{\StringTok}[1]{\textcolor[rgb]{0.31,0.60,0.02}{#1}}
\newcommand{\VariableTok}[1]{\textcolor[rgb]{0.00,0.00,0.00}{#1}}
\newcommand{\VerbatimStringTok}[1]{\textcolor[rgb]{0.31,0.60,0.02}{#1}}
\newcommand{\WarningTok}[1]{\textcolor[rgb]{0.56,0.35,0.01}{\textbf{\textit{#1}}}}
\usepackage{longtable,booktabs}
% Fix footnotes in tables (requires footnote package)
\IfFileExists{footnote.sty}{\usepackage{footnote}\makesavenoteenv{longtable}}{}
\usepackage{graphicx,grffile}
\makeatletter
\def\maxwidth{\ifdim\Gin@nat@width>\linewidth\linewidth\else\Gin@nat@width\fi}
\def\maxheight{\ifdim\Gin@nat@height>\textheight\textheight\else\Gin@nat@height\fi}
\makeatother
% Scale images if necessary, so that they will not overflow the page
% margins by default, and it is still possible to overwrite the defaults
% using explicit options in \includegraphics[width, height, ...]{}
\setkeys{Gin}{width=\maxwidth,height=\maxheight,keepaspectratio}
\setlength{\emergencystretch}{3em}  % prevent overfull lines
\providecommand{\tightlist}{%
  \setlength{\itemsep}{0pt}\setlength{\parskip}{0pt}}
\setcounter{secnumdepth}{5}
% Redefines (sub)paragraphs to behave more like sections
\ifx\paragraph\undefined\else
\let\oldparagraph\paragraph
\renewcommand{\paragraph}[1]{\oldparagraph{#1}\mbox{}}
\fi
\ifx\subparagraph\undefined\else
\let\oldsubparagraph\subparagraph
\renewcommand{\subparagraph}[1]{\oldsubparagraph{#1}\mbox{}}
\fi

% set default figure placement to htbp
\makeatletter
\def\fps@figure{htbp}
\makeatother

\usepackage{booktabs}
\usepackage[]{natbib}
\bibliographystyle{apalike}

\title{Mini-Workshop Panel Data Analysis}
\author{Marko Bachl (mit Material von Michael Scharkow)}
\date{Sommersemester 2020 \textbar{} IJK Hannover}

\begin{document}
\maketitle

{
\setcounter{tocdepth}{1}
\tableofcontents
}
\hypertarget{uxfcberblick}{%
\chapter{Überblick}\label{uxfcberblick}}

\hypertarget{inhalt-des-virtuellen-mini-workshops}{%
\section{Inhalt des virtuellen Mini-Workshops}\label{inhalt-des-virtuellen-mini-workshops}}

\begin{itemize}
\item
  Der Mini-Workshop bietet eine \emph{pragmatische} Einführung in die Analyse von Panel-Daten aus Erhebungen mit mindestens drei Wellen. Konkret liegt der Fokus auf sogenannten \emph{micro panels}, also Datensätzen mit relativ vielen Fällen und relativ wenigen Messzeitpunkten (das klassische Befragungspanel).
\item
  In der Analyse beschränken uns hier auf Varianten der \emph{linearen} Regressionsmodelle. Wir beginnen mit den grundlegenden \emph{fixed effects} und \emph{random effects} Modellen. Dann betrachten wir das \emph{within-between} Modell, das als eine Integration des \emph{fixed effects} Modell in das \emph{random effects} Modell verstanden werden kann. Dies ist auch eine gute Grundlage für den Einstieg in verschiedene Erweiterungen, zum Beispiel zu verallgemeinerten linearen Modellen oder zu Wachstumskurvenmodellen. Diese sind aber nicht Teil dieses Mini-Workshops.
\item
  Wir schätzen die Modelle mit etablierten \emph{least-squares} und \emph{maximum likelihood} Methoden. Gerade bei den \emph{within-between} Modellen sind bayesianische Schätzmethoden, z.B. \emph{MCMC sampling} (implementiert in \href{https://mc-stan.org/}{Stan}), unabhängig von statistisch-philosophischen Überlegungen sehr interessant. Bei Interesse kann ich nur empfehlen, hier einen Einstieg zu finden.
\item
  Zur Aufbereitung der Daten, Visualisierung und Modell-Schätzung verwenden wir \texttt{R} mit dem \texttt{tidyverse} und eine kleine Zahl speziallisierter Pakete für die Modellschätzung. Der Fokus des Workshops liegt aber auf der substantiellen Arbeit mit den Modellen, nicht auf der Umsetzung in \texttt{R}.
\end{itemize}

\hypertarget{welche-inhalte-wir-nicht-behandeln}{%
\section{Welche Inhalte wir nicht behandeln}\label{welche-inhalte-wir-nicht-behandeln}}

\begin{itemize}
\item
  Der Workshop ist kein Statistik- oder Ökonometrie-Kurs. Ich bin --- wie auch ihr --- ausgebildeter Sozialwissenschaftler. Die statistischen Grundlagen, auf denen der Workshop aufbaut, gehen aus den Grundlagentexten \citep{bellExplainingFixedEffects2015, vaiseyWhatYouCan2017} hervor.
\item
  Grundkentnisse in \texttt{R} setze ich voraus, insbesondere Datentransformationen innerhalb des \texttt{tidyverse}. Wir werden aber keine komplizierten Dinge in \texttt{R} tun. Auch ohne weiterführende \texttt{R}-Kenntnisse sollten die Inhalte des Workshops in Bezug auf die datenanalystischen Verfahren klar werden.
\item
  Wir werden nicht viel Zeit auf die verschiedenen Schätzer, deren Effizienz und Bias, die verschiedenen Algorithmen und Datentransformationen verwenden.
\item
  Wir werden keine Beweise oder Ableitungen besprechen. Wir setzen keine Kenntnisse in Matrixalgebra voraus --- weder meiner- noch eurerseits.
\item
  Wir behandeln einen sehr kleinen Ausschnitt möglicher Modelle für Panel-Daten. Der konzentrieren uns auf regressionsbasierte Modelle zur Schätzung kausaler Effekte. Damit behandeln wir insbesondere nicht die vielfälltigen Verfahren, die in einem SEM-Framework verortet sind: längsschnittliche Messmodelle, Prozessmodelle, (random intercept) cross-lagged panel Modelle, Latent State-Trait Modelle, etc. Auch Modelle, in denen die Zeit-Variable als kontinuierlich (z.B. Tag der Erhebung im Gegensatz zu Indikator für Panelwelle) verwendet wird (z.B. Continuous Time Structural Equation Modeling), behandeln wir nicht.
\item
  Fehlende Daten (Panelmortalität, Ausfall von Einheiten in einzelnen Wellen) sind ein großes Thema in der Längsschnittanalyse. Wir werden es hier ignorieren, bis auf den Hinweis, dass alle Fälle, die in mindestens zwei bzw. drei Wellen Daten haben, grundsätzlich Informationen zur Schätzung beitragen.
\end{itemize}

\hypertarget{aufbau-des-workshops}{%
\section{Aufbau des Workshops}\label{aufbau-des-workshops}}

\begin{itemize}
\tightlist
\item
  Inhaltlicher Aufbau: Siehe Kapitel-Gliederung
\end{itemize}

\hypertarget{material}{%
\subsection*{Material}\label{material}}
\addcontentsline{toc}{subsection}{Material}

\begin{itemize}
\item
  Dieses Dokument + R Skripte: (Hoffentlich) mehr oder weniger selbsterklärendes Material

  \begin{itemize}
  \tightlist
  \item
    Kuratierte Form ist dieses HTML-Dokument
  \item
    Es gibt auch ein PDF, das ich aber nicht formatiert habe
  \end{itemize}
\item
  Screencast: Ich gehe über das Material und erkläre es auf der Audio-Spur. Mal sehen, wie hilfreich das ist. Die Screencasts stelle ich über das LMS zur Verfügung.
\item
  Übungen: Zu einigen Analysen gibt es Übungsaufgaben.

  \begin{itemize}
  \tightlist
  \item
    Bei der \emph{Wiederholung} geht es darum, die Modelle leicht zu verändern (durch Anpassen der \texttt{R}-Skripte aus dem Material) und die Ergebnisse der angepassten Modelle zu interpretieren.
  \item
    Bei der \emph{Anwendung} geht es darum, in Anlehung an die Beispiele eigene Modelle zu spezifizieren und diese zu interpretieren.
  \end{itemize}
\end{itemize}

\hypertarget{pakete}{%
\subsection*{Pakete}\label{pakete}}
\addcontentsline{toc}{subsection}{Pakete}

Wir verwenden die folgenden Pakete

\begin{Shaded}
\begin{Highlighting}[]
\ControlFlowTok{if}\NormalTok{ (}\OperatorTok{!}\KeywordTok{require}\NormalTok{(}\StringTok{"pacman"}\NormalTok{)) }\KeywordTok{install.packages}\NormalTok{(}\StringTok{"pacman"}\NormalTok{)}
\NormalTok{pacman}\OperatorTok{::}\KeywordTok{p_load}\NormalTok{(tidyverse, broom, haven)}
\KeywordTok{theme_set}\NormalTok{(}\KeywordTok{theme_bw}\NormalTok{())  }\CommentTok{# ggplot theme}

\KeywordTok{tibble}\NormalTok{(}\DataTypeTok{package =} \KeywordTok{c}\NormalTok{(}\StringTok{"R"}\NormalTok{, }\KeywordTok{sort}\NormalTok{(pacman}\OperatorTok{::}\KeywordTok{p_loaded}\NormalTok{()))) }\OperatorTok\StringTok{ }\KeywordTok{mutate}\NormalTok{(}\DataTypeTok{version =} \KeywordTok{map_chr}\NormalTok{(package, }
    \OperatorTok{~}\KeywordTok{as.character}\NormalTok{(pacman}\OperatorTok{::}\KeywordTok{p_version}\NormalTok{(}\DataTypeTok{package =}\NormalTok{ .x)))) }\OperatorTok\StringTok{ }\NormalTok{knitr}\OperatorTok{::}\KeywordTok{kable}\NormalTok{()}
\end{Highlighting}
\end{Shaded}

\begin{tabular}{l|l}
\hline
package & version\\
\hline
R & 3.6.2\\
\hline
broom & 0.5.4\\
\hline
dplyr & 0.8.4\\
\hline
forcats & 0.4.0\\
\hline
ggplot2 & 3.2.1\\
\hline
haven & 2.2.0\\
\hline
pacman & 0.5.1\\
\hline
purrr & 0.3.3\\
\hline
readr & 1.3.1\\
\hline
stringr & 1.4.0\\
\hline
tibble & 2.1.3\\
\hline
tidyr & 1.0.2\\
\hline
tidyverse & 1.3.0\\
\hline
\end{tabular}

\hypertarget{einfuxfchrung}{%
\chapter{Einführung}\label{einfuxfchrung}}

\hypertarget{luxe4ngsschnittdaten}{%
\section{Längsschnittdaten}\label{luxe4ngsschnittdaten}}

\hypertarget{begriffe}{%
\subsection*{Begriffe}\label{begriffe}}
\addcontentsline{toc}{subsection}{Begriffe}

\begin{itemize}
\tightlist
\item
  Wiederholte Querschnittserhebungen (time series cross sectional, TSCS): \(n\) unabhängige Fälle (repräsentativ für dieselbe Grundgesamtheit) zu mehreren Messzeitpunkten \(t\).
\item
  Zeitreihe: Eine Einheit mit vielen Messzeitpunkten (\(n = 1\), \(t > 30\)).
\item
  Paneldaten: Dieselben Einheiten mit wiederholten Messungen (\(n > 30\), \(t \ge 2\))

  \begin{itemize}
  \tightlist
  \item
    Macro panel: \(n\) klein, \(t\) groß (z.B. jährliche Untersuchung von Staaten, 1950--2015)
  \item
    Micro panel \(n\) groß, \(t\) klein (typisches Befragungspanel)
  \end{itemize}
\item
  In diesem Workshop geht es um \emph{micro panels} mit \(t > 2\)
\end{itemize}

\hypertarget{vorteile-von-paneldaten}{%
\subsection*{Vorteile von Paneldaten}\label{vorteile-von-paneldaten}}
\addcontentsline{toc}{subsection}{Vorteile von Paneldaten}

\begin{itemize}
\tightlist
\item
  Paneldaten erlauben die Identifikation von kausalen Effekten unter schwächeren Annahmen (im Vergleich zu Querschnittsdaten).

  \begin{itemize}
  \tightlist
  \item
    Wir haben einige (aber nicht perfekte!) Informationen über die zeitliche Abfolge von Veränderungen.
  \item
    Wir können untersuchen, ob, und wenn ja, wie ein Ereignis (eine Veränderung eines Prädiktors) das Kriterium verändert.
  \end{itemize}
\item
  Paneldaten erlauben die Untersuchung von individuellen Verläufen
\end{itemize}

\hypertarget{kausale-effekte-mit-paneldaten-schuxe4tzen}{%
\subsection*{Kausale Effekte mit Paneldaten schätzen}\label{kausale-effekte-mit-paneldaten-schuxe4tzen}}
\addcontentsline{toc}{subsection}{Kausale Effekte mit Paneldaten schätzen}

\hypertarget{bedingungen}{%
\subsubsection*{Bedingungen}\label{bedingungen}}
\addcontentsline{toc}{subsubsection}{Bedingungen}

\begin{enumerate}
\def\labelenumi{\arabic{enumi}.}
\tightlist
\item
  Kovariation zwischen \(X\) und \(Y\) (bivariate Korrelation \(r_{XY}\) )
\item
  \(X\) muss logisch vor \(Y\) liegen
\item
  Keine (nicht beobachteten) Störvariablen (kein \(Z\) mit kausalem Effekt auf \(X\) und \(Y\))
\end{enumerate}

\hypertarget{herausforderungen-auch-bzw.-gerade-mit-paneldaten}{%
\subsubsection*{Herausforderungen (auch bzw. gerade mit Paneldaten)}\label{herausforderungen-auch-bzw.-gerade-mit-paneldaten}}
\addcontentsline{toc}{subsubsection}{Herausforderungen (auch bzw. gerade mit Paneldaten)}

\begin{itemize}
\tightlist
\item
  Entsprechung der zeitlichen Entfalltung des Effekts und des Designs (Abstände, Verläufe)
\item
  Reliabilität und Konstruktstabilität

  \begin{itemize}
  \tightlist
  \item
    Reliabilität: Bei geringer Reliabilität beobachten wir Veränderungen, die aber auf Rauschen in der Messung zurückgehen.
  \item
    Konstruktstabilität: Wenn die Messungen über die Zeit ihre Bedeutung verändern, modellieren wir keine Veränderung des latenten Konstrukts von Interesse.
  \end{itemize}
\item
  Panelmortalität und Paneleffekte

  \begin{itemize}
  \tightlist
  \item
    Panelmortalität: Einheiten (Befragte) fallen aus, möglicherweise systematisch mit Bezug auf die Konstrukte oder Effekte, die uns interessieren.
  \item
    Paneleffekte: Einheiten (Befragte) verändern sich durch die Messung (z.B. Lernen von Wissensfragen, Anregung durch Fragen zu Medienangeboten)
  \end{itemize}
\end{itemize}

\hypertarget{format-von-datensuxe4tzen-mit-paneldaten}{%
\subsection*{Format von Datensätzen mit Paneldaten}\label{format-von-datensuxe4tzen-mit-paneldaten}}
\addcontentsline{toc}{subsection}{Format von Datensätzen mit Paneldaten}

\begin{figure}
\centering
\includegraphics{figs/longwide.png}
\caption{\(i\) indiziert Einheiten, \(t\) indiziert Messzeitpunkte, \(y\) ist eine Variable}
\end{figure}

\begin{itemize}
\item
  Die Modelle in diesem Workshop nutzen das \emph{long format}
\item
  Datensätze können von einem ins andere Format transformiert werden, z.B. im \texttt{tidyverse}:

  \begin{itemize}
  \tightlist
  \item
    \texttt{tidyr::gather()} und \texttt{tidyr::spread()} oder
  \item
    \texttt{tidyr::pivot\_longer()} und \texttt{tidyr::pivot\_wider()}
  \end{itemize}
\end{itemize}

\hypertarget{beispiel-daten}{%
\section{Beispiel-Daten}\label{beispiel-daten}}

\begin{itemize}
\item
  Titel: Soziale Normen im alltäglichen Umgang mit den Konsequenzen der Corona-Krise
\item
  sponsored by Jule Scheper und Sophie Bruns
\item
  Thema der Erhebung: Die Corona-Pandemie hat Regierungen auf der ganzen Welt dazu veranlasst, Reglungen zur Reduzierung der raschen Ausbreitung des Virus einzuführen. Die deutsche Bundesregierung hat am 22. März 2020 mehrere Maßnahmen zur Einschränkung sozialer Kontakte beschlossen. Diese Einschränkungen im sozialen Leben sind vollkommen neu und jede*r Einzelne muss sich auf diese Regelungen und die neue Lebenssituation einstellen. Diese Studie beschäftigt sich mit der Frage, wie Menschen sich im Alltag mit der Corona-Pandemie beschäftigen und wie sie mit den Regelungen zur Beschränkung sozialer Kontakte umgehen. Im Mittelpunkt der Untersuchung steht die Entstehung und Veränderung von sozialen Normen und persönlichen Einstellungen zur Beschränkung sozialer Kontakte über die Zeit.
\item
  Im Rahmen des Workshops steht der Einfluss der sozialen Normen und der eigenen Einstellung zum Verhalten auf das tatsächliche Social Distancing-Verhalten im Mittelpunkt.
\item
  Zeitraum der Erhebung: 1.4.-28.4.2020
\item
  Datum der Messzeitpunkte: Die Befragung besteht aus vier Wellen. Jede Welle war für eine Woche im Feld und bezog sich immer auf die vorherige Kalenderwoche.

  \begin{itemize}
  \tightlist
  \item
    Welle 1: Erhebungszeitraum vom 1.4.-7.4., Bezugszeitraum vom 23.3. bis 29.4.
  \item
    Welle 2: Erhebungszeitraum vom 8.4.-14.4., Bezugszeitraum vom 30.3. bis 5.4.
  \item
    Welle 3: Erhebungszeitraum vom 15.4.-21.4., Bezugszeitraum vom 6.4. bis 12.4.
  \item
    Welle 4: Erhebungszeitraum vom 22.4.-28.4., Bezugszeitraum vom 13.4. bis 19.4.
  \end{itemize}
\item
  Nachvollziehen der Aufbereitung in \texttt{R/data.R}
\item
  Direkt laden (z.B. für Übungen) aus \texttt{R/data/data.rds}
\end{itemize}

\begin{tabular}{l|l}
\hline
Variablenname & Label\\
\hline
alter & Alter\\
\hline
besorg1 & Ich bin besorgt, wenn ich an Corona denke.\\
\hline
bildung & Bildungsabschluss\\
\hline
desnormp1 & …sind in der letzten Woche rausgegangen, auch wenn es sich nicht um einen Arztbesuch, Arbeitsweg, Spaziergang/Sport, Einkauf oder Hilfestellungen handelte.\\
\hline
desnormp2 & …haben sich in der letzten Woche in ihrer Freizeit mit mehr als einer anderen Person getroffen, die nicht im gleichen Haushalt lebt.\\
\hline
desnormp3 & …haben in der letzten Woche weniger als 1,5 Meter Abstand zu Personen gehalten, die nicht im gleichen Haushalt leben.\\
\hline
desnormp4 & ...haben sich in der letzten Woche strikt an die Maßnahmen zur Beschränkung sozialer Kontakte gehalten.\\
\hline
ein1 & Ich finde es in Ordnung, wenn man rausgeht, auch wenn es sich nicht um einen Arztbesuch, Arbeitsweg, Spaziergang/Sport, Einkauf oder Hilfe handelt.\\
\hline
ein2 & Ich finde es in Ordnung, wenn man sich in seiner Freizeit mit mehr als einer anderen Person trifft, die nicht im gleichen Haushalt lebt.\\
\hline
ein3 & Ich finde es in Ordnung, wenn man weniger als 1,5 Meter Abstand zu Personen hält, die nicht im gleichen Haushalt leben.\\
\hline
ein4 & Ich finde es wichtig, dass die Empfehlung zur Beschränkung sozialer Kontakte strikt eingehalten werden.\\
\hline
ein5 & Ich finde es richtig, dass generell Abstand gehalten werden soll.\\
\hline
injnormp1 & …finden es in Ordnung, wenn man rausgeht, auch wenn es sich nicht um einen Arztbesuch, Arbeitsweg, Spaziergang/Sport, Einkauf oder Hilfestellungen handelt.\\
\hline
injnormp2 & … finden es in Ordnung, wenn man sich in seiner Freizeit mit mehr als einer anderen Person trifft, die nicht im gleichen Haushalt lebt.\\
\hline
injnormp3 & …finden es in Ordnung, wenn man weniger als 1,5 Meter Abstand zu Personen hält, die nicht im gleichen Haushalt leben.\\
\hline
injnormp4 & ...finden es in Ordnung, wenn man sich strikt an die Maßnahmen zur Beschränkung sozialer Kontakte hält.\\
\hline
kompeer\_s1 & Freunde\\
\hline
kompeer\_s2 & Familie und Partner oder Partnerin\\
\hline
kompeer\_s3 & Bekannte (z.B. Arbeitskollegen und -kolleginnen, Vereinsmitglieder)\\
\hline
kompeer\_s4 & Prominente und/oder Influencer\\
\hline
med1 & Zeitungen \& Zeitschriften (z.B. Die ZEIT, Bild, Focus, der Spiegel)\\
\hline
med2 & Öffentlich-rechtliche Fernsehsender (z.B. ARD, ZDF, h1)\\
\hline
med3 & Private Fernsehsender (z.B. RTL, ProSieben)\\
\hline
med4 & Öffentlich-Rechtliche Radiosender (z.B. DLF, n-joy, NDR)\\
\hline
med5 & Private Radiosender (z.B. 89.0 RTL, ffn)\\
\hline
sex & Geschlecht W4 dummy\\
\hline
stress & Ich fühle mich durch die Corona-Pandemie gestresst.\\
\hline
verh1 & Ich bin rausgegangen, auch wenn es sich nicht um einen Arztbesuch, Arbeitsweg, Einkauf, Spaziergang/Sport oder Hilfestellung gehandelt hat.\\
\hline
verh2 & Ich habe mich mit mehr als einer Person getroffen, die nicht in meinem Haushalt lebt.\\
\hline
verh3 & Ich habe weniger 1,5 Meter Abstand zu Personen gehalten, die nicht in meinem Haushalt leben.\\
\hline
verh4 & Ich habe mich strikt an die Maßnahmen zur Beschränkung sozialer Kontakte gehalten.\\
\hline
verh5 & Ich habe mich im Privaten mit Freunden oder Familienmitgliedern getroffen, die nicht in meinem Haushalt leben.\\
\hline
verh6 & Ich war länger draußen als für einen üblichen Spaziergang (z.B. saß auf der Wiese oder am See).\\
\hline
verhint1 & Rausgehen, auch wenn es sich nicht um einen Arztbesuch, Arbeitsweg, Einkauf, Spaziergang/Sport oder Hilfestellung handelt.\\
\hline
verhint2 & Mich mit mehr als einer Person treffen, die nicht in meinem Haushalt lebt.\\
\hline
verhint3 & Weniger als 1,5 Meter Abstand zu Personen halten, die nicht in meinem Haushalt leben.\\
\hline
verhint4 & Mich strikt an die Maßnahmen zur Beschränkung sozialer Kontakte halten.\\
\hline
verhint5 & Mich im Privaten mit Freunden oder Familienmitgliedern treffen, die nicht in meinem Haushalt leben.\\
\hline
verhint6 & Mich länger draußen aufhalten als für einen üblichen Spaziergang (z.B. auf der Wiese oder am See sitzen).\\
\hline
veruns & Ich bin verunsichert durch die Corona-Krise.\\
\hline
\end{tabular}

\hypertarget{pooled-ols-wrong}{%
\section{Pooled OLS (WRONG!)}\label{pooled-ols-wrong}}

\begin{itemize}
\tightlist
\item
  Als erstes Beispiel wollen wir uns einer klassischen Frage aus der Theory of Planned Behavior zuwenden. Wir interessieren uns für den Effekt der Verhaltensintention auf das (berichtete) Verhalten (schließlich würden wir zum Start des Workshops ja gerne etwas finden ;)). Konkret betrachten wir den Effekt des Vorhabens, entgegen der Empfehlungen ohne relevanten Grund die Wohnung zu verlassen, auf den Selbstbericht, dies auch zu tun. Die beiden relevanten Variablen sind \texttt{verh1} und \texttt{verhint1}. Die Abbildung zeigt ihre Entwicklung über die vier Wellen für 10 zufällig ausgewählte Personen.
\end{itemize}

\begin{Shaded}
\begin{Highlighting}[]
\NormalTok{id_smple =}\StringTok{ }\KeywordTok{sample}\NormalTok{(}\KeywordTok{unique}\NormalTok{(d}\OperatorTok{$}\NormalTok{IDsosci), }\DecValTok{10}\NormalTok{)}

\NormalTok{d }\OperatorTok\StringTok{ }\KeywordTok{filter}\NormalTok{(IDsosci }\OperatorTok\StringTok{ }\NormalTok{id_smple) }\OperatorTok\StringTok{ }\KeywordTok{select}\NormalTok{(IDsosci, wave, verh1, verhint1) }\OperatorTok\StringTok{ }
\StringTok{    }\KeywordTok{gather}\NormalTok{(variable, value, }\OperatorTok{-}\NormalTok{IDsosci, }\OperatorTok{-}\NormalTok{wave) }\OperatorTok\StringTok{ }\KeywordTok{ggplot}\NormalTok{(}\KeywordTok{aes}\NormalTok{(wave, value, }\DataTypeTok{group =}\NormalTok{ IDsosci, }
    \DataTypeTok{color =}\NormalTok{ IDsosci)) }\OperatorTok{+}\StringTok{ }\KeywordTok{geom_line}\NormalTok{(}\DataTypeTok{position =} \KeywordTok{position_jitter}\NormalTok{(}\DataTypeTok{height =} \FloatTok{0.2}\NormalTok{), }\DataTypeTok{show.legend =} \OtherTok{FALSE}\NormalTok{) }\OperatorTok{+}\StringTok{ }
\StringTok{    }\KeywordTok{facet_wrap}\NormalTok{(}\StringTok{"variable"}\NormalTok{)}
\end{Highlighting}
\end{Shaded}

\includegraphics{workshop_panel_files/figure-latex/vis-ex1-1.pdf}

\begin{itemize}
\tightlist
\item
  Das einfachste Modell, diesen Effekt zu schätzen, ist eine einfache OLS Regression der Verhaltensintention auf das Verhalten.
\end{itemize}

\begin{Shaded}
\begin{Highlighting}[]
\KeywordTok{lm}\NormalTok{(verh1 }\OperatorTok{~}\StringTok{ }\NormalTok{verhint1, }\DataTypeTok{data =}\NormalTok{ d) }\OperatorTok\StringTok{ }\KeywordTok{tidy}\NormalTok{() }\OperatorTok\StringTok{ }\KeywordTok{mutate_if}\NormalTok{(is.numeric, round, }\DecValTok{2}\NormalTok{)}
\end{Highlighting}
\end{Shaded}

\begin{verbatim}
## # A tibble: 2 x 5
##   term        estimate std.error statistic p.value
##   <chr>          <dbl>     <dbl>     <dbl>   <dbl>
## 1 (Intercept)     0.47      0.02      19.6       0
## 2 verhint1        0.59      0.01      53.8       0
\end{verbatim}

\begin{itemize}
\tightlist
\item
  Das Modell besagt, dass die Häufigkeit, nach raus zu gehen, mit jedem Punkt auf der Intentionsskala um ca. \(b_{verhint1} = 0.6\) Punkte steigt.
\end{itemize}

\hypertarget{warum-ist-pooled-ols-immer-falsch-statistische-theorie}{%
\subsection*{Warum ist Pooled OLS immer falsch? Statistische Theorie}\label{warum-ist-pooled-ols-immer-falsch-statistische-theorie}}
\addcontentsline{toc}{subsection}{Warum ist Pooled OLS immer falsch? Statistische Theorie}

\begin{itemize}
\tightlist
\item
  Wir nennen dieses Modell \emph{pooled} OLS, da alle Beobachtungen einfach zusammengeworfen werden, ohne zu beachten, dass einige von ihnen zusammen gehören, da sie von denselben Personen stammen.
\end{itemize}

\begin{enumerate}
\def\labelenumi{\arabic{enumi})}
\tightlist
\item
  Exogenitätsannahme ist verletzt, \(E(u_i|x_i) \neq 0\), da

  \begin{itemize}
  \tightlist
  \item
    Korrelationen zwischen den Variablen \(x\) gehen auf nicht gemessene Eigenschaften der Einheiten zurück, z.B. Eigenschaften der Person \(z_i\), die sowohl \(x_i\) als auch \(y_i\) beeinflussen.
  \item
    Auch bekannt als \emph{omitted variable bias}
  \item
    Könnte behoben werden, wenn alle \(z_i\) im Modell wären; diese Idee wird später wichtig
  \end{itemize}
\item
  Annahmen Homoskedastizität und unkorrelierte Residuen sind (wahrscheinlich) verletzt

  \begin{itemize}
  \tightlist
  \item
    Systematische Variation der Residuen zwischen Einheiten
  \item
    Wahrscheinlich serielle Korreationen durch die zeitliche Abhängigkeit der Messungen
  \end{itemize}
\item
  Annahme der Unabhängigkeit der Bebobachtungen verletzt

  \begin{itemize}
  \tightlist
  \item
    Überschätzung der Information von abhängigen Fällen (dieselbe Information ist mehrmals im Datensatz)

    \begin{itemize}
    \tightlist
    \item
      Zu kleine Standardfehler, zu große Zahl der Freiheitsgrade in Signifikanz-Tests
    \end{itemize}
  \item
    Die wahre Fallzahl (effective sample size) ist kleiner als Zahl der Zeilen im Datensatz (\emph{long format})
  \end{itemize}
\end{enumerate}

\hypertarget{warum-ist-pooled-ols-immer-falsch-inhaltliche-uxfcberlegungen}{%
\subsection*{Warum ist pooled OLS immer falsch? Inhaltliche Überlegungen}\label{warum-ist-pooled-ols-immer-falsch-inhaltliche-uxfcberlegungen}}
\addcontentsline{toc}{subsection}{Warum ist pooled OLS immer falsch? Inhaltliche Überlegungen}

\begin{itemize}
\tightlist
\item
  Unser Ziel ist es, den wahren kausalen Effekt von \(X\) auf \(Y\) zu schätzen.
\item
  Pooled OLS vermischt aber zwei Quellen von Unterschieden in den Daten: Den (kausalen) Effekt innerhalb der Personen (within) und die Unterschiede zwischen Personen (between).
\item
  Within und between Effekte können sich in Größe und sogar in der Richtung unterscheiden!
\item
  Die Schätzung aus einem poold OLS Modell vermischt den kausalen Effekt und die interindividuellen Unterschiede.
\item
  In der Sprache von Interventionsstudien ist das ein Selbstselektions-Problem: Was passiert, wenn Personen, die vor dem Treatment \(x\) schon höhere Werte in \(y\) haben?
\item
  Außerdem fällt auf, dass im einfachen OLS Modell nichts darauf hindeutet, dass es sich um Paneldaten handelt. Selbst wenn wir die genannten Probleme nicht hätten, hätten wir auch nichts durch die Paneldaten gewonnen.
\end{itemize}

\hypertarget{pooled-ols-within-und-between---eine-illustration}{%
\subsection*{Pooled OLS, within und between - eine Illustration}\label{pooled-ols-within-und-between---eine-illustration}}
\addcontentsline{toc}{subsection}{Pooled OLS, within und between - eine Illustration}

\begin{itemize}
\item
  Zum Abschluss noch ein imagninäres Beispiel, um den Unterschied von intraindividuellen (within) Effekten und interindividuellen Unterschieden zu verdeutlichen. Wir führen eine Panel-Studie mit acht Personen und sechs Messzeitpunkten zum Zusammenhang von Bier-Konsum und Hangover durch. Wir interessieren uns für die kausale Frage, ob mehr Bier zu einem schlimmeren Kater führt.
\item
  In der pooled OLS Analyse wird einfach die Rergressionsgerade durch alle Beobachung gelegt. Es zeigt sich ein negativer Zusammenhang. Je mehr Bier konsumiert wurde, desto schwächer fällt der Hangover aus.
\end{itemize}

\includegraphics{workshop_panel_files/figure-latex/unnamed-chunk-8-1.pdf}

\begin{itemize}
\tightlist
\item
  Wenn wir aber für alle acht Personen separat den Zusammenhang zwischen Bierkonsum und Kater berechnen (so genanntes no pooling Modell), ergibt sich ein anderes Bild. Für alle Personen gilt mehr oder weniger deutlich: Je mehr Bier konsumiert wurde, desto stärker fällt der Hangover aus (within).
\end{itemize}

\includegraphics{workshop_panel_files/figure-latex/unnamed-chunk-9-1.pdf}

\begin{itemize}
\tightlist
\item
  Dazu kommt ein systematischer Unterschied zwischen den Personen (between): Personen, die im Durchschnitt mehr Bier trinken, haben im Durchschnitt einen schwächeren Hangover. Dies könnte auf eine nicht beobachte Drittvariable auf Ebene der Personen zurück gehen:

  \begin{itemize}
  \tightlist
  \item
    Vielleicht trinken Personen, die wissen, dass sie nicht so anfällig für einen Hangover sind, mehr, während Personen, die immer einen starken Kater haben, schon aus Angst vor dem nächsten Tag weniger trinken.
  \item
    Oder es ist ein Gewöhnungseffekt: Personen, die häufig viel trinken, gewöhnen sich an den Kater und nehmen ihn als weniger schlimm wahr. Oder mit Lemmy: ``A kid once said to me ``Do you get hangovers?'' I said, ``To get hangovers you have to stop drinking.''
  \end{itemize}
\item
  Mit den vorliegenden Daten können wir die Frage nach dem Prozess nicht beantworten, da wir die Drittvariable nicht gemessen haben. Wir können aber \emph{alle} Variablen kontrollieren, die auf Personenebene liegen, z.B., indem wir wie in der Abbildung für jede Person ein separates Modell schätzen. Dann können Unterschiede zwischen den Einheiten per Modelldefinition keinen Einfluss auf die Schätzung haben. Etwas ähnliches passiert im \emph{fixed effects} Modell, das wir im nächsten Abschnitt besprechen.
\end{itemize}

\hypertarget{fixed-effects-modelle}{%
\chapter{Fixed effects Modelle}\label{fixed-effects-modelle}}

\hypertarget{konzeptionelle-einfuxfchrung}{%
\section{Konzeptionelle Einführung}\label{konzeptionelle-einfuxfchrung}}

\begin{itemize}
\tightlist
\item
  Im ersten Teil des Abschnitts zu \emph{fixed effects} Modellen beschäftigen wir uns mit den Grundlagen der Modellierung. Dazu nutzen wir \texttt{stats::lm()} (übliche OLS-Schätzung linearer Modelle in \texttt{R}).
\end{itemize}

\hypertarget{wie-kuxf6nnen-wir-den-kauselen-within-person-effekt-mit-paneldaten-schuxe4tzen}{%
\subsection*{Wie können wir den kauselen (within-person) Effekt mit Paneldaten schätzen?}\label{wie-kuxf6nnen-wir-den-kauselen-within-person-effekt-mit-paneldaten-schuxe4tzen}}
\addcontentsline{toc}{subsection}{Wie können wir den kauselen (within-person) Effekt mit Paneldaten schätzen?}

\begin{enumerate}
\def\labelenumi{\arabic{enumi})}
\tightlist
\item
  Separate OLS Modelle für jede Person schätzen und Koeffizeinten mitteln (no pooling).
\item
  Alle \(X\) und \(Y\) Variablen um die Mittelwerte der Person zentrieren (within transformation).
\item
  Dummy-Variablen für jede Person in das Regressionsmodell aufnehmen (least squares dummy variables {[}LSDV{]} estimation).
\end{enumerate}

\begin{itemize}
\tightlist
\item
  Alle drei Varianten entfernen die (beobachteten und nicht beobachten,) über die Zeit konstanten Unterschiede zwischen den Personen.
\item
  Varianten 2 und 3 entsprechen dem klassischen \emph{fixed effects} Modell. Die Unterschiede zwischen den Personen werden kontrolliert, indem die personenspezifischen Mittelwerte vor der Schätzung entfernt werden (2) oder für jede Person im Modell geschätzt werden (3).

  \begin{itemize}
  \tightlist
  \item
    \(y_{it}-\bar{y_{i}} = (x_{it} - \bar{x_{i}})'\beta + (u_{it} - \bar{u_{i}})\) oder \(y_{it} = \beta' x_{it}' + \alpha_i + u_{it}\)
  \end{itemize}
\item
  In Variante 1 dürfen die kausalen within-person Effekte zwischen den Personen variieren. Unter der Annahme homogener Treatment-Effekte (entspricht der typischen Annahme im randomisierten Between-Subject-Experiment) entspricht das Ergebnis asymptotisch den Varianten 2 und 3.

  \begin{itemize}
  \tightlist
  \item
    Der Schätzer ist aber weniger effizient, da zufällige Unterschiede in den Effekten zwischen den Personen aufgegriffen werden.
  \item
    Im letzten Teil des Abschnitts zum within-between-Modell kommen wir auf diesen Punkt zurück, wenn wir die Annahme homogener Treatment-Effekte lockern.
  \end{itemize}
\end{itemize}

\hypertarget{no-pooling}{%
\subsection*{No pooling}\label{no-pooling}}
\addcontentsline{toc}{subsection}{No pooling}

\begin{Shaded}
\begin{Highlighting}[]
\NormalTok{d }\OperatorTok\StringTok{ }\KeywordTok{group_by}\NormalTok{(IDsosci) }\OperatorTok\StringTok{ }\CommentTok{# mutate(chk = sd(verh1) != 0 & sd(verhint1) != 0) %>% filter(chk) %>%}
\KeywordTok{nest}\NormalTok{() }\OperatorTok\StringTok{ }\KeywordTok{mutate}\NormalTok{(}\DataTypeTok{mdls =} \KeywordTok{map}\NormalTok{(data, }\OperatorTok{~}\KeywordTok{tidy}\NormalTok{(}\KeywordTok{lm}\NormalTok{(verh1 }\OperatorTok{~}\StringTok{ }\NormalTok{verhint1, }\DataTypeTok{data =}\NormalTok{ .x)))) }\OperatorTok\StringTok{ }\KeywordTok{unnest}\NormalTok{(mdls) }\OperatorTok\StringTok{ }
\StringTok{    }\KeywordTok{ungroup}\NormalTok{() }\OperatorTok\StringTok{ }\KeywordTok{select}\NormalTok{(}\OperatorTok{-}\NormalTok{data) }\OperatorTok\StringTok{ }\KeywordTok{na.omit}\NormalTok{() }\OperatorTok\StringTok{ }\KeywordTok{filter}\NormalTok{(statistic }\OperatorTok{!=}\StringTok{ }\OtherTok{Inf}\NormalTok{) }\OperatorTok\StringTok{ }\KeywordTok{filter}\NormalTok{(term }\OperatorTok{==}\StringTok{ }
\StringTok{    "verhint1"}\NormalTok{) }\OperatorTok\StringTok{ }\KeywordTok{mutate_if}\NormalTok{(is.numeric, round, }\DecValTok{2}\NormalTok{) }\OperatorTok\StringTok{ }\NormalTok{print }\OperatorTok\StringTok{ }\KeywordTok{summarise}\NormalTok{(}\DataTypeTok{estimate =} \KeywordTok{mean}\NormalTok{(estimate), }
    \DataTypeTok{std.error =} \KeywordTok{sqrt}\NormalTok{(}\KeywordTok{mean}\NormalTok{(std.error}\OperatorTok{^}\DecValTok{2}\NormalTok{)))  }\CommentTok{# simple approximation}
\end{Highlighting}
\end{Shaded}

\begin{verbatim}
## # A tibble: 236 x 6
##    IDsosci term     estimate std.error statistic p.value
##    <chr>   <chr>       <dbl>     <dbl>     <dbl>   <dbl>
##  1 050IPY  verhint1     1.25      0.56  2.24e+ 0    0.15
##  2 05J4R8  verhint1     0.45      0.18  2.50e+ 0    0.13
##  3 08BDZJ  verhint1     0.33      0.53  6.30e- 1    0.59
##  4 0EO9L2  verhint1     1.67      0.67  2.50e+ 0    0.13
##  5 0F5L9Z  verhint1     0         0.71  0.          1   
##  6 0KYYAJ  verhint1     0.45      0.18  2.50e+ 0    0.13
##  7 0ONV4O  verhint1     1         0     9.01e+15    0   
##  8 0Q5XIM  verhint1    -0.27      0.31 -8.70e- 1    0.48
##  9 0ZCKB5  verhint1    -0.35      0.5  -6.90e- 1    0.56
## 10 114OWA  verhint1     0.33      0.33  1.00e+ 0    0.42
## # ... with 226 more rows
\end{verbatim}

\begin{verbatim}
## # A tibble: 1 x 2
##   estimate std.error
##      <dbl>     <dbl>
## 1    0.493     0.520
\end{verbatim}

\begin{itemize}
\tightlist
\item
  Wir erhalten für jede Person einen Schätzer mit Standardfehler. Wir können diese mitteln, um einen Schätzer des durchschnittlichen kausalen Effekts zu erhalten.
\item
  Wir müssen die Schätzer entfernen, bei denen das Modell wegen eines perfekten Zusammenhangs oder wegen fehlender \emph{within-person} Varianz keine OLS Lösung hat.
\end{itemize}

\hypertarget{within-transformation}{%
\subsection*{Within Transformation}\label{within-transformation}}
\addcontentsline{toc}{subsection}{Within Transformation}

\begin{Shaded}
\begin{Highlighting}[]
\NormalTok{d_wi =}\StringTok{ }\NormalTok{d }\OperatorTok\StringTok{ }\KeywordTok{select}\NormalTok{(IDsosci, verh1, verhint1) }\OperatorTok\StringTok{ }\KeywordTok{group_by}\NormalTok{(IDsosci) }\OperatorTok\StringTok{ }\KeywordTok{mutate}\NormalTok{(}\DataTypeTok{verh1_wi =}\NormalTok{ verh1 }\OperatorTok{-}\StringTok{ }
\StringTok{    }\KeywordTok{mean}\NormalTok{(verh1), }\DataTypeTok{verhint1_wi =}\NormalTok{ verhint1 }\OperatorTok{-}\StringTok{ }\KeywordTok{mean}\NormalTok{(verhint1)) }\OperatorTok\StringTok{ }\KeywordTok{ungroup}\NormalTok{()}

\NormalTok{d_wi }\OperatorTok\StringTok{ }\KeywordTok{select}\NormalTok{(}\OperatorTok{-}\NormalTok{IDsosci) }\OperatorTok\StringTok{ }\NormalTok{summary}
\end{Highlighting}
\end{Shaded}

\begin{verbatim}
##      verh1          verhint1      verh1_wi      verhint1_wi   
##  Min.   :1.000   Min.   :1.0   Min.   :-3.00   Min.   :-3.00  
##  1st Qu.:1.000   1st Qu.:1.0   1st Qu.:-0.25   1st Qu.:-0.25  
##  Median :1.000   Median :1.0   Median : 0.00   Median : 0.00  
##  Mean   :1.526   Mean   :1.8   Mean   : 0.00   Mean   : 0.00  
##  3rd Qu.:2.000   3rd Qu.:2.0   3rd Qu.: 0.00   3rd Qu.: 0.00  
##  Max.   :5.000   Max.   :5.0   Max.   : 3.00   Max.   : 3.00
\end{verbatim}

\begin{Shaded}
\begin{Highlighting}[]
\NormalTok{d_wi }\OperatorTok\StringTok{ }\KeywordTok{lm}\NormalTok{(verh1_wi }\OperatorTok{~}\StringTok{ }\NormalTok{verhint1_wi, }\DataTypeTok{data =}\NormalTok{ .) }\OperatorTok\StringTok{ }\KeywordTok{tidy}\NormalTok{() }\OperatorTok\StringTok{ }\KeywordTok{mutate_if}\NormalTok{(is.numeric, }
\NormalTok{    round, }\DecValTok{2}\NormalTok{)}
\end{Highlighting}
\end{Shaded}

\begin{verbatim}
## # A tibble: 2 x 5
##   term        estimate std.error statistic p.value
##   <chr>          <dbl>     <dbl>     <dbl>   <dbl>
## 1 (Intercept)     0         0.01       0         1
## 2 verhint1_wi     0.34      0.01      24.7       0
\end{verbatim}

\begin{itemize}
\tightlist
\item
  Intuitive Interpretation: Eine Abweichung vom Personen-Durchschnitt in \(X\) um einen Punkt führt zu einer Abweichung vom Personen-Durchschnitt in \(Y\) um \(b_{X}\) Punkte.
\item
  Hier: Wenn eine Person um einen Punkt wahrscheinlicher rausgehen möchte als üblich, dann wird sie 0.34 Punkte häufiger rausgehen (beides auf 5er Skalen).
\item
  Das ist durchaus ein bedeutsamer Effekt. Aber zur Erinnerung: Der naiven pooled OLS Schätzung zufolge war der Effekt fast doppelt so groß.
\end{itemize}

\hypertarget{least-squares-mit-dummy-variablen-lsdv}{%
\subsection*{Least Squares mit Dummy Variablen (LSDV)}\label{least-squares-mit-dummy-variablen-lsdv}}
\addcontentsline{toc}{subsection}{Least Squares mit Dummy Variablen (LSDV)}

\begin{Shaded}
\begin{Highlighting}[]
\NormalTok{d }\OperatorTok\StringTok{ }\KeywordTok{lm}\NormalTok{(verh1 }\OperatorTok{~}\StringTok{ }\NormalTok{verhint1 }\OperatorTok{+}\StringTok{ }\KeywordTok{factor}\NormalTok{(IDsosci), }\DataTypeTok{data =}\NormalTok{ .) }\OperatorTok\StringTok{ }\KeywordTok{tidy}\NormalTok{() }\OperatorTok\StringTok{ }\KeywordTok{mutate_if}\NormalTok{(is.numeric, }
\NormalTok{    round, }\DecValTok{2}\NormalTok{) }\OperatorTok\StringTok{ }\KeywordTok{print}\NormalTok{(}\DataTypeTok{n =} \DecValTok{17}\NormalTok{)}
\end{Highlighting}
\end{Shaded}

\begin{verbatim}
## # A tibble: 586 x 5
##    term                  estimate std.error statistic p.value
##    <chr>                    <dbl>     <dbl>     <dbl>   <dbl>
##  1 (Intercept)              0.66       0.28      2.32    0.02
##  2 verhint1                 0.34       0.02     21.4     0   
##  3 factor(IDsosci)02E6C8   -0.34       0.4      -0.85    0.39
##  4 factor(IDsosci)050IPY    1.23       0.4       3.04    0   
##  5 factor(IDsosci)05J4R8    0.32       0.4       0.81    0.42
##  6 factor(IDsosci)08BDZJ    0.98       0.4       2.42    0.02
##  7 factor(IDsosci)0BHGLF    0.570      0.4       1.43    0.15
##  8 factor(IDsosci)0EB6C1    0          0.4       0       1   
##  9 factor(IDsosci)0EO9L2    1.97       0.4       4.87    0   
## 10 factor(IDsosci)0F5L9Z    1.65       0.4       4.09    0   
## 11 factor(IDsosci)0KAKHF    2.22       0.4       5.49    0   
## 12 factor(IDsosci)0KYYAJ   -0.01       0.4      -0.01    0.99
## 13 factor(IDsosci)0ONV4O    0.33       0.4       0.82    0.41
## 14 factor(IDsosci)0PKFWT   -0.09       0.4      -0.21    0.83
## 15 factor(IDsosci)0Q5XIM    0.32       0.4       0.81    0.42
## 16 factor(IDsosci)0ZCKB5    0.32       0.4       0.81    0.42
## 17 factor(IDsosci)114OWA    0.33       0.4       0.82    0.41
## # ... with 569 more rows
\end{verbatim}

\begin{itemize}
\tightlist
\item
  Der Punktschätzer \(b_{X}\) entspricht genau dem Punktschätzer nach der within-person Transformation.
\item
  Zusätzlich gibt die Regressionskonstante den Mittelwert für Person 1 an und die \(n - 1\) Koeffizienten der Dummy-Variablen die Abweichung der übrigen Personen von diesem Mittelwert. Es gelten die üblichen Regeln für die Interpretation solcher Koeffizienten.
\end{itemize}

\hypertarget{welche-modellspezifikation-soll-ich-nutzen}{%
\subsection*{Welche Modellspezifikation soll ich nutzen?}\label{welche-modellspezifikation-soll-ich-nutzen}}
\addcontentsline{toc}{subsection}{Welche Modellspezifikation soll ich nutzen?}

\begin{enumerate}
\def\labelenumi{\arabic{enumi})}
\item
  Der Schätzer des durchschnittlichen kausalen Effekts in der no pooling Spezifikation ist im Vergleich zu den beiden anderen Varianten weniger effizient. Außerdem ist er praktisch schwieriger zu ermitteln, da er erst aus den Schätzern der Einzel-Modelle berechnet werden muss. Wenn wir die Annahme eines homogenen kausalen Effekts treffen (und das tun wir üblicherweise), dann gibt es keinen Grund, das no pooling Modell in der Praxis zu verwenden.
\item
  Die Spezifikationen mit within-person Transformation und LSDV ergeben dieselben Punktschätzer für den kausalen Effekt und sind insofern austauschbar.
\item
  Die Standardfehler des Modells mit einer naiven within-person Transformation (wie oben dargestellt) sind zu klein, da wir die Stichprobenmittelwerte und nicht die (mit Unsicherheit behafteten) Schätzer der Populationsmittelwerte zur Zentrierung verwenden. Die Standardfehler müssen daher angepasst werden (passiert in spezialisierten Software-Paketen automatisch).
\item
  Die LSDV Spezifikation ist in fast jedem Softwarepaket einfach umzusetzen. Mit großen Datensätzen wird aber die Schätzung langsam und der Output unübersichtlich.
\end{enumerate}

\begin{itemize}
\item
  Unabhängig von der Spezifikation gelten weiterhin alle Annahmen der (OLS) Regression. Besonders gern vergessen wird der \emph{omitted variable bias} durch nicht gemessene, über die Zeit variierdene \(Z\). \emph{Fixed effects} Modelle kontrollieren nur die \(Z\), die auf konstante Merkmale der als \emph{fixed effects} spezifizierten Einheiten zurückgehen.
\item
  Insgesamt sind viele quantitative Sozialforscher (v.a. die mit einer Ökonometrie-Ausbildung) der Ansicht, dass \emph{fixed effects} Modelle die beste Methode sind, um kausale Effekte aus nicht-experimentellen Daten zu schätzen.
\end{itemize}

\hypertarget{mehre-fixed-effects-in-einem-modell---perioden-effekte}{%
\subsection{Mehre fixed effects in einem Modell - Perioden-Effekte}\label{mehre-fixed-effects-in-einem-modell---perioden-effekte}}

\begin{itemize}
\tightlist
\item
  Grundsätzlich können in einem Modell beliebig viele \emph{fixed effects} spezifiziert werden.
\item
  In Paneldaten ist der Erhebungszeitpunkt bzw. die Ergebungsperiode (Panelwelle) eine typische Variable, über die verschiedene, für alle Personen konstante Effekte kontrolliert werden können.
\item
  Einige Lehrbücher empfehlen, dies \emph{immer} zu tun, da kausale Effekte von Ereignissen, die für alle Einheiten konstant sind, statistisch nicht identifiziert sind.
\item
  Eine typische Spezifikation ist die Aufnahme eines \emph{fixed effects} für den Indikator der Panelwelle.
\item
  In der LSDV-Spezifikation kann einfach ein weiterer Dummy-Faktor hinzugefügt werden. Die within-person Transformation ist mathematisch komplizierter, wird aber in spezialisierten Software-Pakten im Hintergrund erledigt. Es können auch beide Spezifikationen kombiniert werden, wenn z.B. die Periodeneffekte von inhaltlichem Interesse sind und im Output angezeigt werden sollen (siehe nächsten Teilabschnitt).
\end{itemize}

\hypertarget{ein-beispiel-mit-fixed-effects-fuxfcr-personen-und-perioden}{%
\subsection*{\texorpdfstring{Ein Beispiel mit \emph{fixed effects} für Personen und Perioden}{Ein Beispiel mit fixed effects für Personen und Perioden}}\label{ein-beispiel-mit-fixed-effects-fuxfcr-personen-und-perioden}}
\addcontentsline{toc}{subsection}{Ein Beispiel mit \emph{fixed effects} für Personen und Perioden}

\begin{Shaded}
\begin{Highlighting}[]
\NormalTok{d }\OperatorTok\StringTok{ }\KeywordTok{lm}\NormalTok{(verh1 }\OperatorTok{~}\StringTok{ }\NormalTok{verhint1 }\OperatorTok{+}\StringTok{ }\KeywordTok{factor}\NormalTok{(wave) }\OperatorTok{+}\StringTok{ }\KeywordTok{factor}\NormalTok{(IDsosci), }\DataTypeTok{data =}\NormalTok{ .) }\OperatorTok\StringTok{ }\KeywordTok{tidy}\NormalTok{() }\OperatorTok\StringTok{ }
\StringTok{    }\KeywordTok{mutate_if}\NormalTok{(is.numeric, round, }\DecValTok{2}\NormalTok{) }\OperatorTok\StringTok{ }\KeywordTok{print}\NormalTok{(}\DataTypeTok{n =} \DecValTok{17}\NormalTok{)}
\end{Highlighting}
\end{Shaded}

\begin{verbatim}
## # A tibble: 589 x 5
##    term                  estimate std.error statistic p.value
##    <chr>                    <dbl>     <dbl>     <dbl>   <dbl>
##  1 (Intercept)               0.61      0.28      2.15    0.03
##  2 verhint1                  0.32      0.02     19.6     0   
##  3 factor(wave)2             0.02      0.03      0.66    0.51
##  4 factor(wave)3             0.14      0.03      4.25    0   
##  5 factor(wave)4             0.12      0.03      3.53    0   
##  6 factor(IDsosci)02E6C8    -0.32      0.4      -0.81    0.42
##  7 factor(IDsosci)050IPY     1.28      0.4       3.19    0   
##  8 factor(IDsosci)05J4R8     0.35      0.4       0.87    0.39
##  9 factor(IDsosci)08BDZJ     1.03      0.4       2.57    0.01
## 10 factor(IDsosci)0BHGLF     0.6       0.4       1.5     0.13
## 11 factor(IDsosci)0EB6C1     0         0.4       0       1   
## 12 factor(IDsosci)0EO9L2     2.04      0.4       5.06    0   
## 13 factor(IDsosci)0F5L9Z     1.69      0.4       4.23    0   
## 14 factor(IDsosci)0KAKHF     2.29      0.4       5.68    0   
## 15 factor(IDsosci)0KYYAJ     0.01      0.4       0.02    0.99
## 16 factor(IDsosci)0ONV4O     0.34      0.4       0.85    0.4 
## 17 factor(IDsosci)0PKFWT    -0.08      0.4      -0.2     0.84
## # ... with 572 more rows
\end{verbatim}

\begin{itemize}
\tightlist
\item
  \(b_{verhint1}\) quantifiziert weiterhin den kausalen Effekt von Interesse. Er ist robust gegen die Kontrolle des Periodeneffekts.
\item
  Die \(b_{wave_t}\) zeigen den Kontrast zur ersten Welle. In diesem Fall sind liegen in der dritten und vierten Welle die Häufigkeiten des Rausgehens höher als noch in den ersten beiden Wellen.
\item
  Die \(b_{id_i}\) zeigen weiterhin den Kontrast zu Person 1 (substantiell nicht sonderlich interessant).
\end{itemize}

\hypertarget{uxfcbungsaufgaben-zur-grundsuxe4tzlichen-spezifikation-von-fixed-effects-modellen}{%
\subsection*{\texorpdfstring{Übungsaufgaben zur grundsätzlichen Spezifikation von \emph{fixed effects} Modellen}{Übungsaufgaben zur grundsätzlichen Spezifikation von fixed effects Modellen}}\label{uxfcbungsaufgaben-zur-grundsuxe4tzlichen-spezifikation-von-fixed-effects-modellen}}
\addcontentsline{toc}{subsection}{Übungsaufgaben zur grundsätzlichen Spezifikation von \emph{fixed effects} Modellen}

\begin{enumerate}
\def\labelenumi{\arabic{enumi})}
\tightlist
\item
  Schätze den kausalen Effekt von X auf Y mit einem \emph{fixed effects} Modell (andere Beispiele aus dem Datensatz).

  \begin{itemize}
  \tightlist
  \item
    INHALTLICH FORMULIERTE FRAGE
  \item
    Vergleiche die Modelle mit und ohne Periodeneffekte.
  \end{itemize}
\item
  Spezifiziere, schätze und interpretiere ein eigenes \emph{fixed effects} Modell mit dem Beipsieldatensatz.
\end{enumerate}

\hypertarget{fixed-effects-modelle-in-der-praktischen-anwendung}{%
\section{Fixed effects Modelle in der praktischen Anwendung}\label{fixed-effects-modelle-in-der-praktischen-anwendung}}

\begin{itemize}
\tightlist
\item
  mit \texttt{plm}
\end{itemize}

\hypertarget{random-effects-models}{%
\chapter{Random effects models}\label{random-effects-models}}

XXX

\hypertarget{within-between-models}{%
\chapter{Within-between models}\label{within-between-models}}

XXX

\bibliography{book.bib,packages.bib,references.bib}

\end{document}
