\PassOptionsToPackage{unicode=true}{hyperref} % options for packages loaded elsewhere
\PassOptionsToPackage{hyphens}{url}
%
\documentclass[]{book}
\usepackage{lmodern}
\usepackage{amssymb,amsmath}
\usepackage{ifxetex,ifluatex}
\usepackage{fixltx2e} % provides \textsubscript
\ifnum 0\ifxetex 1\fi\ifluatex 1\fi=0 % if pdftex
  \usepackage[T1]{fontenc}
  \usepackage[utf8]{inputenc}
  \usepackage{textcomp} % provides euro and other symbols
\else % if luatex or xelatex
  \usepackage{unicode-math}
  \defaultfontfeatures{Ligatures=TeX,Scale=MatchLowercase}
\fi
% use upquote if available, for straight quotes in verbatim environments
\IfFileExists{upquote.sty}{\usepackage{upquote}}{}
% use microtype if available
\IfFileExists{microtype.sty}{%
\usepackage[]{microtype}
\UseMicrotypeSet[protrusion]{basicmath} % disable protrusion for tt fonts
}{}
\IfFileExists{parskip.sty}{%
\usepackage{parskip}
}{% else
\setlength{\parindent}{0pt}
\setlength{\parskip}{6pt plus 2pt minus 1pt}
}
\usepackage{hyperref}
\hypersetup{
            pdftitle={Mini-Workshop Panel Data Analysis},
            pdfauthor={Marko Bachl (mit Material von Michael Scharkow)},
            pdfborder={0 0 0},
            breaklinks=true}
\urlstyle{same}  % don't use monospace font for urls
\usepackage{longtable,booktabs}
% Fix footnotes in tables (requires footnote package)
\IfFileExists{footnote.sty}{\usepackage{footnote}\makesavenoteenv{longtable}}{}
\usepackage{graphicx,grffile}
\makeatletter
\def\maxwidth{\ifdim\Gin@nat@width>\linewidth\linewidth\else\Gin@nat@width\fi}
\def\maxheight{\ifdim\Gin@nat@height>\textheight\textheight\else\Gin@nat@height\fi}
\makeatother
% Scale images if necessary, so that they will not overflow the page
% margins by default, and it is still possible to overwrite the defaults
% using explicit options in \includegraphics[width, height, ...]{}
\setkeys{Gin}{width=\maxwidth,height=\maxheight,keepaspectratio}
\setlength{\emergencystretch}{3em}  % prevent overfull lines
\providecommand{\tightlist}{%
  \setlength{\itemsep}{0pt}\setlength{\parskip}{0pt}}
\setcounter{secnumdepth}{5}
% Redefines (sub)paragraphs to behave more like sections
\ifx\paragraph\undefined\else
\let\oldparagraph\paragraph
\renewcommand{\paragraph}[1]{\oldparagraph{#1}\mbox{}}
\fi
\ifx\subparagraph\undefined\else
\let\oldsubparagraph\subparagraph
\renewcommand{\subparagraph}[1]{\oldsubparagraph{#1}\mbox{}}
\fi

% set default figure placement to htbp
\makeatletter
\def\fps@figure{htbp}
\makeatother

\usepackage{booktabs}
\usepackage[]{natbib}
\bibliographystyle{apalike}

\title{Mini-Workshop Panel Data Analysis}
\author{Marko Bachl (mit Material von Michael Scharkow)}
\date{Sommersemester 2020 \textbar{} IJK Hannover}

\begin{document}
\maketitle

{
\setcounter{tocdepth}{1}
\tableofcontents
}
\hypertarget{uxfcberblick}{%
\chapter{Überblick}\label{uxfcberblick}}

\hypertarget{inhalt-des-virtuellen-mini-workshops}{%
\section{Inhalt des virtuellen Mini-Workshops}\label{inhalt-des-virtuellen-mini-workshops}}

\begin{itemize}
\item
  Der Mini-Workshop bietet eine \emph{pragmatische} Einführung in die Analyse von Panel-Daten aus Erhebungen mit mindestens drei Wellen. Konkret liegt der Fokus auf sogenannten \emph{micro panels}, also Datensätzen mit relativ vielen Fällen und relativ wenigen Messzeitpunkten (das klassische Befragungspanel).
\item
  In der Analyse beschränken uns hier auf Varianten der \emph{linearen} Regressionsmodelle. Wir beginnen mit den grundlgenden \emph{fixed effects} und \emph{random effects} Modellen. Dann betrachten wir das \emph{within-between} Modell, das als eine Integration des \emph{fixed effects} Modell in das \emph{random effects} Modell verstanden werden kann. Dies ist auch eine gute Grundlage für den Einstieg in verschiedene Erweiterungen, zum Beispiel zu verallgemeinerten linearen Modellen oder zu Wachstumskurvenmodellen. Diese sind aber nicht Teil dieses Mini-Workshops.
\item
  Wir schätzen die Modelle mit etablierten \emph{least-squares} und \emph{maximum likelihood} Methoden. Gerade bei den \emph{within-between} Modellen sind bayesianische Schätzmethoden, z.B. \emph{MCMC sampling} (implementiert in \href{https://mc-stan.org/}{Stan}), unabhängig von den statistisch-philosophischen, sehr interessant. Bei Interesse kann ich nur empfehlen, hier einen Einstieg zu finden.
\item
  Zur Aufbereitung der Daten, Visualisierung und Modell-Schätzung verwenden wir \texttt{R} mit dem \texttt{tidyverse} und eine kleine Zahl speziallisierter Pakete für die Modellschätzung. Der Fokus des Workshops liegt aber auf der substantiellen Arbeit mit den Modellen, nicht auf der Umsetzung in \texttt{R}.
\end{itemize}

\hypertarget{welche-inhalte-wir-nicht-behandeln}{%
\section{Welche Inhalte wir nicht behandeln}\label{welche-inhalte-wir-nicht-behandeln}}

\begin{itemize}
\item
  Der Workshop ist kein Statistik- oder Ökonometrie-Kurs. Ich bin --- wie auch ihr --- ausgebildeter Sozialwissenschaftler. Die statistischen Grundlagen, auf denen der Workshop aufbaut, geht aus den Grundlagentexten \citep{bellExplainingFixedEffects2015, vaiseyWhatYouCan2017} hervor.
\item
  Grundkentnisse in \texttt{R} setze ich voraus, insbesondere Datentransformationen innerhalb des \texttt{tidyverse}.
\item
  Wir werden nicht viel Zeit auf die verschiedenen Schätzer, deren Effizienz und Bias, die verschiedenen Algorithmen und Datentransformationen verwenden.
\item
  Wir werden keine Beweise oder Ableitungen besprechen. Wir setzen keine Kenntnisse in Matrixalgebra voraus --- weder meiner- noch eurerseits.
\item
  Wir behandeln einen sehr kleinen Ausschnitt möglicher Modelle für Panel-Daten. Der konzentrieren uns auf regressionsbasierte Modelle zur Schätzung kausaler Effekte. Damit behandeln wir insbesondere nicht die vielfälltigen Verfahren, die in einem SEM-Framework verortet sind: längsschnittliche Messmodelle, Prozessmodelle, (random intercept) cross-lagged panel Modelle, Latent State-Trait Modelle, etc.
\end{itemize}

\hypertarget{aufbau-des-workshops}{%
\section{Aufbau des Workshops}\label{aufbau-des-workshops}}

\begin{itemize}
\tightlist
\item
  Inhaltlicher Aufbau: Siehe Kapitel-Gliederung
\end{itemize}

\hypertarget{material}{%
\subsection{Material}\label{material}}

\begin{itemize}
\item
  Dieses Dokument + R Skripte: (Hoffentlich) mehr oder weniger selbsterklärendes Material

  \begin{itemize}
  \tightlist
  \item
    Kuratierte Form ist dieses HTML-Dokument
  \item
    Es gibt auch ein PDF, das ich aber nicht formatiert habe
  \end{itemize}
\item
  Screencast: Ich gehe über das Material und erkläre es auf der Audio-Spur. Mal sehen, wie hilfreich das ist. Die Screencasts stelle ich über das LMS zur Verfügung.
\item
  Übungen: Zu einigen Analysen gibt es Übungsaufgaben.

  \begin{itemize}
  \tightlist
  \item
    Bei der \emph{Wiederholung} geht es darum, die Modelle leicht zu verändern (durch Anpassen der \texttt{R}-Skripte aus dem Material) und die Ergebnisse der angepassten Modelle zu interpretieren.
  \item
    Bei der \emph{Anwendung} geht es darum, in Anlehung an die Beispiele eigene Modelle zu spezifizieren und diese zu interpretieren.
  \end{itemize}
\end{itemize}

\hypertarget{intro}{%
\chapter{Einführung}\label{intro}}

XXX

\hypertarget{fixed-effects-models}{%
\chapter{Fixed effects models}\label{fixed-effects-models}}

XXX

\hypertarget{random-effects-models}{%
\chapter{Random effects models}\label{random-effects-models}}

XXX

\hypertarget{within-between-models}{%
\chapter{Within-between models}\label{within-between-models}}

XXX

\bibliography{book.bib,packages.bib,references.bib}

\end{document}
